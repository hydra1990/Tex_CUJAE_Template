\addtocontents{toc}{\vspace{1em}} % Añadir un espacio en los contenidos, por estética

\lhead{\emph{Introducción}}  % Establecer el encabezado superior izquierdo a "Introducción"
% Para que no le ponga delante a la Introducción la palabra Capítulo
\chapter*{Introducción}
\addcontentsline{toc}{chapter}{Introducción}

\paragraph{}Lorem ipsum dolor sit amet, consectetur adipiscing elit, sed do eiusmod tempor incididunt ut labore et dolore magna aliqua. Ut enim ad minim veniam, quis nostrud exercitation ullamco laboris nisi ut aliquip ex ea commodo consequat. Duis aute irure dolor in reprehenderit in voluptate velit esse cillum dolore eu fugiat nulla pariatur. Excepteur sint occaecat cupidatat non proident, sunt in culpa qui officia deserunt mollit anim id est laborum \cite{noauthor_something_2007}.

Curabitur pretium tincidunt lacus. Nulla gravida orci a odio. Nullam varius, turpis et commodo pharetra, est eros bibendum elit, nec luctus magna felis sollicitudin mauris. Integer in mauris eu nibh euismod gravida. Duis ac tellus et risus vulputate vehicula. Donec lobortis risus a elit. Etiam tempor. Ut ullamcorper, ligula eu tempor congue, eros est euismod turpis, id tincidunt sapien risus a quam. Maecenas fermentum consequat mi. Donec fermentum. Pellentesque malesuada nulla a mi. Duis sapien sem, aliquet nec, commodo eget, consequat quis, neque. Aliquam faucibus, elit ut dictum aliquet, felis nisl adipiscing sapien, sed malesuada diam lacus eget erat. Cras mollis scelerisque nunc. Nullam arcu. Aliquam consequat. Curabitur augue lorem, dapibus quis, laoreet et, pretium ac, nisi. Aenean magna nisl, mollis quis, molestie eu, feugiat in, orci. In hac habitasse platea dictumst.

\subsubsection*{Situación Problemática:}
\paragraph{}Lorem ipsum dolor sit amet, consectetur adipiscing elit, sed do eiusmod tempor incididunt ut labore et dolore magna aliqua. Ut enim ad minim veniam, quis nostrud exercitation ullamco laboris nisi ut aliquip ex ea commodo consequat. Duis aute irure dolor in reprehenderit in voluptate velit esse cillum dolore eu fugiat nulla pariatur. Excepteur sint occaecat cupidatat non proident, sunt in culpa qui officia deserunt mollit anim id est laborum.

\subsubsection*{Problema a resolver:}
\paragraph{}¿Lorem ipsum dolor sit amet, consectetur adipiscing elit, sed do eiusmod tempor incididunt ut labore et dolore magna aliqua. Ut enim ad minim veniam, quis nostrud exercitation ullamco laboris nisi ut aliquip ex ea commodo consequat. Duis aute irure dolor in reprehenderit in voluptate velit esse cillum dolore eu fugiat nulla pariatur. Excepteur sint occaecat cupidatat non proident, sunt in culpa qui officia deserunt mollit anim id est laborum?
\paragraph{}El presente trabajo tiene como \textbf{objeto de estudio} en el \textbf{escenario del cliente/usuario}, ..... En el \textbf{escenario de la informática} se centra en .....
\paragraph{}Dentro del \textbf{campo de acción} en el \textbf{escenario del cliente/usuario} se encuentran ..... En el \textbf{escenario de la informática} se centra en .....
\paragraph{}Para darle solución a la problemática planteada se define como \textbf{objetivo general}, del presente trabajo de diploma, .....
\paragraph{}Para dar cumplimiento al objetivo general, será necesario dar cumplimiento a los \textbf{objetivos específicos} que se desglosan a continuación:
\begin{enumerate}
	\item Lorem ipsum dolor sit amet, consectetur adipiscing elit, sed do eiusmod tempor incididunt ut labore et dolore magna aliqua
\end{enumerate}

\paragraph{}Para dar cumplimiento a los diferentes objetivos específicos se establecieron las siguientes \textbf{tareas de investigación} por objetivo:

\textbf{Objetivo 1 ....}
\begin{enumerate}
	\item Hacer algo.

\subsubsection*{Valor Práctico}
\paragraph{}Con el desarrollo del sistema propuesto se espera, como aporte práctico: .....
\paragraph{}Para dar cumplimiento al objetivo propuesto el documento se estructura en cinco capítulos: \CapUno, \CapDos, \CapTres, \CapCuatro, \CapCinco. Cada cap\'{i}tulo está estructurado en introducción, desarrollo y conclusiones parciales.
\paragraph{Capítulo 1: \CapUno.} Explicación del capítulo 1.
\paragraph{Capítulo 2: \CapDos.} Explicación del capítulo 2. 
\paragraph{Capítulo 3: \CapTres.} Explicación del capítulo 3.
\paragraph{Capítulo 4: \CapCuatro.} Explicación del capítulo 4.
\paragraph{Capítulo 5: \CapCinco.} Explicación del capítulo 5.
\paragraph{}Por último, se presentan las conclusiones, recomendaciones, referencias bibliográficas, así como un glosario de términos que ayuda en el entendimiento de los conceptos específicos de la investigación.