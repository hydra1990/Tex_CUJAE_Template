% Para todos los capítulos, use la funcionalidad nueva chap{} en vez de chapter{}
% Esto hará que el texto en la izquierda superior de la página sea el mismo que el del capítulo

% Los Nombres de los capítulos se definen en Configuraciones/Administrativo.tex
\chap{\CapCinco}

\sf
\section{Introducción}
\paragraph{}En el desarrollo de un proyecto de software es necesario practicar pruebas para descubrir errores en el funcionamiento del sistema. Las pruebas pueden clasificarse en: validación, funcionales, caja negra, caja blanca y automatizadas \cite{pittet_distintos_nodate}. La necesidad de determinar la factibilidad económica de un proyecto de software es importante para determinar el costo beneficio de su desarrollo. En este tipo de industria el cálculo se realiza sobre la base de la determinación del tiempo estimado para el desarrollo del proyecto, los costos asociados y los beneficios tangibles e intangibles. En este capítulo se diseñan los casos de prueba utilizados en el desarrollo del sistema, así como la planificación del proyecto a partir de los casos de uso.

\section{Tipos de pruebas ejecutadas y justificación de su selección}
\paragraph{}Las pruebas son un instrumento importante en el proceso de desarrollo de \textit{software}, debido a que evalúan la calidad del producto de \textit{software}, detectan defectos y los documentan, validan los requerimientos planteados y su funcionamiento. Las pruebas tienen el objetivo de detectar la mayor cantidad de errores posible y solucionarlos \cite{pittet_distintos_nodate}.

\section{Tipos de pruebas existentes}
Existen varios tipos de prueba a aplicar al \textit{software} dentro de las que se encuentran, según muestra la tabla \ref{table:TiposPruebas} \cite{quijano_que_2018}.

\begin{table}[ht]
	\begin{tabular}{ |p{3cm}|p{2cm}|p{4cm}|p{4cm}|p{2cm}| }
		\hline
		\textbf{Tipo de prueba}
		& \textbf{Practicada sobre}
		& \textbf{Técnicas}
		& \textbf{Resultados}
		& \textbf{Realizadas por} \\ \hline
		
		\textbf{Validación}
		& Software
		& Evaluaciones, inspecciones y tutoriales
		& Comprobar que lo que se ha especificado coincide con lo que el usuario quería.
		& Probador \\ \hline
		
		\textbf{Funcionales}
		& Software
		& Ejecución, revisión y retroalimentación de funcionalidades. Se realizan sobre el sistema en funcionamiento
		& Comprobar que se cumple con lo especificado a través de casos de uso
		& Probador \\ \hline
		
		\textbf{Caja negra}
		& Software
		& Valores de entradas y salidas
		& Se desconoce la implementación del código por el probador
		& Probador \\ \hline
		
		\textbf{Caja blanca}
		& Software
		& Verificación de  interacción entre componentes que implementan un caso de uso
		& Interacción interna entre los componentes del sistema. Ligado al código fuente
		& Probador \\ \hline
		
		\textbf{Automatizadas}
		& Software
		& Automatizan actividades comunes que no requieren inteligencia humana
		& Rapidez, confiabilidad, programable
		& Analista \\
		
		\hline
	\end{tabular}
	\caption[Tipos de pruebas existentes a aplicar al software]{Tipos de pruebas existentes a aplicar al software \cite{pittet_distintos_nodate}}
	\label{table:TiposPruebas}
\end{table}

\paragraph{}Para probar el \textit{software} se aplicó la variante de caja negra, ya que se lleva a cabo sobre la interfaz del \textit{software}, al proporcionar al sistema valores de entrada para medir el comportamiento esperado de las salidas, se mide si el resultado obtenido es el esperado \cite{pittet_distintos_nodate}.
\paragraph{}En este caso se realizaron pruebas durante una semana a tres de los casos de uso del sistema.

\section{Diseño de los casos de prueba}
Caso de uso de prueba introducir saldos iniciales

Tipo de Prueba Realizada: Caja Negra

Condiciones de ejecución: Los importes límites de pagos menores tienen que estar presentes

\begin{table}[H]
	\sf
	\begin{supertabular}{|p{4cm}|p{4cm}|p{4cm}|p{4cm}|}
		\hline
		
		\multicolumn{2}{l}{CU Introducir saldos iniciales}
		&  \hspace{2 mm}
		&  \hspace{2 mm}
		\\ \hline
		
		\textbf{Nombre del Escenario}
		& \textbf{Descripción del escenario de la prueba}
		& \textbf{Respuesta del sistema}
		& Flujo central \\ \hline
		
		EC 1.1 Introducir saldos satisfactoriamente
		& Los datos son cargados por la aplicación al inicio de la misma
		& 4. El sistema carga los datos del último arqueo de caja como saldos iniciales
		& 1. Abrir la aplicación
		
		2. Ir a la opción Introducir saldos iniciales
		
		3. Teclear los datos del último arqueo impreso \\ \hline
		
		EC 1.2 Fallo en la carga de datos
		& Los datos no se cargan en la aplicación debido a que su suma excede del valor asignado a la caja
		& 4. El sistema muestra un mensaje de error con el texto "El Importe desglosado excede el máximo permitido"
		& 1. Abrir la aplicación
		
		2. Ir a la opción Introducir saldos iniciales
		
		3. Teclear los datos del último arqueo
		 \\
				
		\hline
	\end{supertabular}
	\caption[Caso de uso de prueba introducir saldos iniciales]{Caso de uso de prueba introducir saldos iniciales}
	\label{table:CUP_IntrodSaldos}
\end{table}

Caso de uso de prueba Emitir Modelo de Vale para Pagos Menores

Tipo de Prueba Realizada: Caja Negra

Condiciones de ejecución: Tiene que estar identificado el usuario en la aplicación

\begin{table}[H]
	\sf
	\begin{supertabular}{|p{4cm}|p{4cm}|p{4cm}|p{4cm}|}
		\hline
		
		\multicolumn{2}{l}{CU Emitir Vales para Pagos Menores}
		&  \hspace{2 mm}
		&  \hspace{2 mm}
		\\ \hline
		
		\textbf{Nombre del Escenario}
		& \textbf{Descripción del escenario de la prueba}
		& \textbf{Respuesta del sistema}
		& Flujo central \\ \hline
		
EC 1.1 Emitir modelo de vale para pagos menores
		& Los datos del vale se muestran por la aplicación en un formulario
		& 4. El sistema muestra un formulario de vale para pagos menores
		
		6. El sistema muestra el vale activo
		& 1. Abrir la aplicación
		
		2. Ir al menú Operaciones
		
		3. Seleccionar Vale para pagos menores
		
		5. Introducir el monto a extraer
		 \\ \hline
		
		EC 1.2 Emitir modelo de vale para pagos menores
		& Los datos del vale se muestran por la aplicación en una planilla cuando se carga una cantidad de efectivo no disponible
		& 4. El sistema muestra un formulario de vale para pagos menores
		
		6. El sistema muestra un mensaje de error con la cantidad de la denominación existente
		& 1. Abrir la aplicación
		
		2. Ir al menú Reportes
		
		3. Seleccionar la opción Vale para pagos menores
		
		5. Se teclea por denominación e importe la cantidad objeto de pago superior a la existente en el sistema
		\\
		
		\hline
	\end{supertabular}
	\caption[Caso de uso de prueba emitir modelo de vale para pagos menores]{Caso de uso de prueba emitir modelo de vale para pagos menores}
	\label{table:CUP_ModeloVPM}
\end{table}

Caso de uso de prueba Emitir Recibo de Efectivo

Tipo de Prueba Realizada: Caja Negra

Condiciones de ejecución: Tiene que estar identificado el usuario en la aplicación

\begin{table}[H]
	\sf
	\begin{supertabular}{|p{4cm}|p{4cm}|p{4cm}|p{4cm}|}
		\hline
		
		\multicolumn{2}{l}{CU Emitir Recibo de Efectivo}
		&  \hspace{2 mm}
		&  \hspace{2 mm}
		\\ \hline
		
		\textbf{Nombre del Escenario}
		& \textbf{Descripción del escenario de la prueba}
		& \textbf{Respuesta del sistema}
		& Flujo central \\ \hline
		
		EC 1.1 Emitir modelo de Recibo de Efectivo
		& Los datos del recibo se muestran por la aplicación en un formulario
		& 5. El sistema muestra un modelo de Recibo de Efectivo
		
		6. El sistema muestra un Recibo de Efectivo activo
		
		& 1. Abrir la aplicación
		
		2. Ir al menú Operaciones
		
		3. Seleccionar la opción Recibo de Efectivo
		
		5. El usuario desglosa la cantidad que entregará
		\\ \hline
		
		EC 1.2 Imprimir modelo de Recibo de Efectivo
		& Los datos del recibo se muestran para su impresión
		& 4. El sistema muestra un modelo de Recibo de Efectivo listo para su llenado
		
		6. El sistema muestra la opción de imprimir el recibo
		& 1. Abrir la aplicación
		
		2. Ir al menú Operaciones
		
		3. Seleccionar la opción Recibo de Efectivo
		
		5. Teclear los datos del recibo
		\\ \hline
		
		EC 1.3 Fallo en la impresora
		& El recibo no se puede imprimir debido a que no hay una impresora conectada a la computadora
		& 6. El sistema muestra un modelo de Recibo de Efectivo listo para su impresión
		
		7. El sistema muestra la opción de imprimir el recibo
		& 1. Abrir la aplicación
		
		2. Ir al menú Reportes
		
		3. Seleccionar la opción Recibo de Efectivo
		
		5. Teclear los datos del recibo
		\\
		
		\hline
	\end{supertabular}
	\caption[Caso de uso de prueba Emitir Recibo de Efectivo]{Caso de uso de prueba Emitir Recibo de Efectivo}
	\label{table:CUP_ERE}
\end{table}

Caso de uso de prueba Visualizar Reportes

Tipo de Prueba Realizada: Caja Negra

Condiciones de ejecución: Debe haber al menos una impresora configurada

\begin{table}[H]
	\sf
	\begin{supertabular}{|p{4cm}|p{4cm}|p{4cm}|p{4cm}|}
		\hline
		
		\multicolumn{2}{l}{CU Visualizar reportes}
		&  \hspace{2 mm}
		&  \hspace{2 mm}
		\\ \hline
		
		\textbf{Nombre del Escenario}
		& \textbf{Descripción del escenario de la prueba}
		& \textbf{Respuesta del sistema}
		& Flujo central \\ \hline
		
		EC 1.1 Visualizar reportes
		& Los reportes no se muestran, ni se guardan por la aplicación ni se imprimen
		& 5. El sistema muestra un mensaje de error con el texto que aparece en la figura
		
		6. El menú se cierra sin obtenerse el reporte
		
		& 1. Abrir la aplicación
		
		2. Ir a la opción Reportes
		
		3. Escoger el reporte a imprimir
		
		5. Dar clic a botón aceptar
		\\
		
		\hline
	\end{supertabular}
	\caption[Caso de uso de prueba emitir Recibo de Efectivo]{Caso de uso de prueba emitir Recibo de Efectivo}
	\label{table:CUP_VisRep}
\end{table}

\begin{figure}[H] %Con el paquete float, se pone la opción [H] para que te salga la imagen donde la quieres
	\centering
	\includegraphics[width=15cm,height=3cm]{Figuras/EscenarioCasoPrueba.png}
	\caption{Escenario 1.1 Caso de prueba visualizar reportes}
	\label{fig:EscenarioCasoPrueba}
\end{figure}

\paragraph{}Las pruebas realizadas permitieron verificar que se cumplieron los objetivos propuestos al inicio de la implementación y corregir los errores. Se realizaron dos iteraciones de pruebas, la primera iteración produjo seis errores que se desglosan a continuación:
\begin{enumerate}
	\item Caso de uso de prueba: Introducir Saldos Iniciales
	\begin{enumerate}
		\item El sistema no muestra la información de todas las denominaciones y tiene errores en el cálculo del importe.
		\item El sistema no muestra el mensaje de error.
	\end{enumerate}
	\item Caso de uso de prueba: Emitir Vale para Pagos Menores
	\begin{enumerate}
		\item El formulario muestra errores
		\item El sistema no muestra correctamente la cantidad
	\end{enumerate}
	\item Caso de uso de prueba: Emitir Recibo de Efectivo
	\begin{enumerate}
		\item El sistema presenta errores en el formulario
		\item El sistema no muestra la opción de imprimir
	\end{enumerate}
\end{enumerate}

\paragraph{}Estos errores fueron corregidos mediante la modificación de código fuente en la aplicación.

\section{Esfuerzo del Implementador del sistema}
\begin{table}[ht]
	\begin{tabular}{ |p{4cm}|p{2.5cm}|p{1.5cm}|p{3cm}|p{2cm}| }
		\hline
		\textbf{Tareas Desarrolladas}
		& \textbf{Cantidad Trabajador}
		& \textbf{Salario / Hora}
		& \textbf{Tiempo en Horas Consumidas}
		& \textbf{Total} \\ \hline
		
		Captura de requisitos
		& 1 & 4.25 & 48 h (6 días) & $\$204.00$ \\ \hline
		
		Diseño de la solución propuesta
		& 1 & 4.25 & 200 h (25 días) & $\$850.00$ \\ \hline
		
		Propuesta al usuario
		& 1 & 4.25 & 48 h (6 días) & $\$204.00$ \\ \hline
		
		Selección y estudio de las herramientas
		& 1 & 4.25 & 192 h (24 días) & $\$816.00$ \\ \hline
		
		Implementación y documentación
		& 1 & 4.25 & 1920 h (240 días) & $\$8,160.00$ \\ \hline
		
		Implantación y capacitación
		& 1 & 4.25 & 40 h (5 días) & $\$170.00$ \\ \hline
		
		Pruebas
		& 1 & 4.25 & 80 h (10 días) & $\$340.00$ \\ \hline
		
		\multicolumn{3}{r}{\textbf{Total}} \vline
		& 2,528 horas & $\$10,744.00$ \\
		
		\hline
	\end{tabular}
	\caption[Esfuerzo del Implementador del sistema]{Esfuerzo del Implementador del sistema}
	\label{table:esfuerzoImpl}
\end{table}

\paragraph{}En total se invirtieron en el proyecto 2,528 horas que son equivalentes a 11 meses, tiempo en que se desarrolló el proyecto y la documentación.
\begin{center}
	{ \fboxsep 12pt
		\fcolorbox {black}{white}{
			\begin{minipage}[t]{12cm}
				Estimando que:
				
				Gastos Directos = $\$$ $10,744.00$
				
				Gastos Indirectos = $\$$ $3.00$ por hora (Estimación gastos indirectos consumo diario de electricidad, depreciación medios de cómputos, conexión a internet, papel y otros insumos)
				
				2,528 horas x 3 = $\$$ $843.00$
				
				Total Calculado = $\$$ $11,587.00$ $CUP$
			\end{minipage}
		} }
\end{center}

%\section{Planificación basada en casos de uso}
%\paragraph{}La planificación mediante el análisis de Puntos de Casos de Uso se trata de un método de estimación del tiempo de desarrollo de un proyecto mediante la asignación de “pesos” a un cierto número de factores que lo afectan[33].
%
%\subsection{Cálculo de los puntos de casos de uso sin ajustar.}
%\paragraph{}El primer paso para realizar la planificación es el cálculo de los puntos de los casos de uso sin ajustar, para ello se utilizará la siguiente fórmula:
%
%\begin{center}
%	{ \fboxsep 12pt
%		\fcolorbox {black}{white}{
%			\begin{minipage}[t]{4cm}
%				$PCU = FPA + PCU$
%			\end{minipage}
%		} }
%\end{center}
%\paragraph{}Donde:  
%\begin{itemize}
%	\item $PCU$: Puntos de Casos de Uso sin ajustar.
%	\item $FPA$: Factor de Peso de los Actores sin ajustar.
%	\item $FPCU$: Factor de Peso de los Casos de Uso sin ajustar.
%\end{itemize}
%
%\subsection{Factor de Peso de los Actores sin ajustar}
%\paragraph{}Este valor es resultante de un análisis de los actores presentes en el sistema y su complejidad, que se establece de acuerdo a los criterios que indica la tabla \ref{table:FactorPeso}.
%\paragraph{}El sistema cuenta con 5 actores complejos: Administrador, Usuario, Cajero, Jefe de Departamento y Especialista en Contabilidad; los que cumplen con la clasificación de complejo. De la información anterior se obtiene el Factor de peso x Actores ($FPA$).
%
%\begin{table}[ht]
%	\begin{tabular}{ |p{2.5cm}|p{8cm}|p{1.5cm}|p{1.5cm}|p{1.5cm}| }
%		\hline
%		\textbf{Tipo de actor}
%		& \textbf{Descripción}
%		& \textbf{Factor de peso}
%		& \textbf{Actores}
%		& \textbf{Total} \\ \hline
%		
%		\textbf{Simple}
%		& Otro sistema que interactúa con el sistema a desarrollar mediante una interfaz de aplicación
%		& 1 & 0 & 0 \\ \hline
%		
%		\textbf{Medio}
%		& Otro sistema que interactúa con el sistema a desarrollar mediante un protocolo o una interfaz basada en texto
%		& 2 & 0 & 0 \\ \hline
%		
%		\textbf{Complejo}
%		& Una persona que interactúa con el sistema a desarrollar mediante una interfaz gráfica
%		& 3 & 6 & 18 \\
%		
%		\hline
%	\end{tabular}
%	\caption[Factor de Peso de los actores sin ajustar]{Factor de Peso de los actores sin ajustar}
%	\label{table:FactorPeso}
%\end{table}
%
%\begin{center}
%	{ \fboxsep 12pt
%		\fcolorbox {black}{white}{
%			\begin{minipage}[t]{6cm}
%				$FPA = \sum(FactorDePeso * Actores)$ \\
%				$FPA = 3 * 6$ \\
%				$FPA = 18$
%			\end{minipage}
%		} }
%\end{center}
%
%\subsection{$FPCU$: Factor de Peso de los Casos de Uso sin ajustar} 
%\paragraph{}El valor del factor de peso de los casos de uso sin ajustar es el resultado del análisis de la cantidad de Casos de Uso existentes en el sistema y la complejidad que estos tienen.
%\paragraph{}Para determinar una transacción se tiene en cuenta la secuencia atómica de actividades, las cuales se realizan completamente o no se realiza ninguna.
%\paragraph{}En la tabla \ref{table:FactorPesoCU} se muestra la determinación del Factor de Peso de los Casos de Uso sin ajustar.
%
%\begin{table}[ht]
%	\begin{tabular}{ | l | l | l | l | l | }
%		\hline
%		\textbf{Tipo de caso de uso}
%		& \textbf{Descripción}
%		& \textbf{Peso}
%		& \textbf{\#CU}
%		& \textbf{\#CU*Peso} \\ \hline
%		
%		\textbf{Simple}
%		& El caso de uso contiene de 1 a 3 transacciones
%		& 5 & 26 & 130 \\ \hline
%		
%		\textbf{Medio}
%		& El caso de uso contiene de 4 a 7 transacciones
%		& 10 & 6 & 60 \\ \hline
%		
%		\textbf{Complejo}
%		& El caso de uso contiene de 8 o más transacciones
%		& 15 & 0 & 0 \\
%		
%		\hline
%	\end{tabular}
%	\caption[Factor de Peso de los Casos de Uso sin ajustar]{Factor de Peso de los Casos de Uso sin ajustar}
%	\label{table:FactorPesoCU}
%\end{table}
%
%\begin{center}
%	{ \fboxsep 12pt
%		\fcolorbox {black}{white}{
%			\begin{minipage}[t]{5cm}
%				$FCPU = \sum(\#CU * Peso)$ \\
%				$FCPU = 190$
%			\end{minipage}
%		} }
%\end{center}
%
%\begin{center}
%	{ \fboxsep 12pt
%		\fcolorbox {black}{white}{
%			\begin{minipage}[t]{7cm}
%				De lo anteriormente tenemos: \\
%				$FPA=18$ \\
%				y \\
%				$FCPU = 190$ \\
%				entonces: \\
%				$PCU= FPA + FPCU = 18 + 190 = 208$ \\
%				O sea: \\
%				$PCU = 208$
%			\end{minipage}
%		} }
%\end{center}
%
%\section{Cálculo de puntos de casos de uso ajustados}
%\paragraph{}Una vez que se tenga cálculo de los puntos de casos de uso sin ajustar ($PCU$) se procede a calcular los puntos de casos de uso ajustados ($PCUA$) cuyo resultado es el resultado es el producto del Factor de Complejidad Técnica ($FCT$) por el Factor Ambiente ($FA$), como se muestra a continuación:
%\begin{center}
%	{ \fboxsep 12pt
%		\fcolorbox {black}{white}{
%			\begin{minipage}[t]{5cm}
%				$FCUA = PCU * FCT * FA$
%			\end{minipage}
%		} }
%\end{center}
%\paragraph{Factor de Complejidad Técnica:}“Este valor se estima mediante la cuantificación del peso de un grupo de factores que determinan la complejidad técnica del software. A cada factor se le asigna un valor de 0 a 5 de acuerdo con la relevancia, donde 0 significa un aporte irrelevante y 5 un aporte muy importante” [33].
%
%\begin{table}[ht]
%	\begin{tabular}{ | l |p{4cm}| l | l |p{2cm}|p{6cm}| }
%		\hline
%		\textbf{Factor}
%		& \textbf{Descripción}
%		& \textbf{Peso}
%		& \textbf{Valor}
%		& \textbf{Peso*Valor}
%		& \textbf{Comentario} \\ \hline
%		
%		\textbf{T1}
%		& Sistema distribuido
%		& 2 & 3 & 63
%		& El sistema para desarrollar exige conexión de red, y un servidor de base de datos \\ \hline
%		
%		\textbf{T2}
%		& Objetivos de performance o tiempo de respuesta
%		& 1 & 3 & 3
%		& El sistema tiene un tiempo de respuesta normal \\ \hline
%		
%		\textbf{T3}
%		& Eficiencia del usuario final
%		& 1 & 2 & 2
%		& Existen escasas restricciones de eficiencias en el usuario final \\ \hline
%		
%		\textbf{T4}
%		& Procesamiento interno complejo
%		& 1 & 2 & 2
%		& No existe un procesamiento complejo \\ \hline
%		
%		\textbf{T5}
%		& El código debe ser reutilizable
%		& 1 & 5 & 5
%		& Se desea que el código sea lo más reusable posible por las magnitudes que puede alcanzar el software \\ \hline
%		
%		\textbf{T6}
%		& Facilidad de instalación
%		& 0.5 & 2 & 1
%		& No requiere grandes complejidades para su instalación \\ \hline
%		
%		\textbf{T7}
%		& Facilidad de uso
%		& 0.5 & 5 & 2.5
%		& El sistema presenta una interfaz fácil de usar y es sencilla \\ \hline
%		
%		\textbf{T8}
%		& Portabilidad
%		& 2 & 5 & 10
%		& El sistema es multiplataforma \\ \hline
%		
%		\textbf{T9}
%		& Facilidad de cambio
%		& 1 & 4 & 4
%		& El sistema está diseñado por capas por lo que los cambios se pueden realizar de forma sencilla \\ \hline
%		
%		\textbf{T10}
%		& Concurrencia
%		& 1 & 3 & 3
%		& Pueden acceder varios usuarios a la vez \\ \hline
%		
%		\textbf{T11}
%		& Incluye objetivos especiales de seguridad
%		& 1 & 3 & 3
%		& La seguridad se trata a partir de los privilegios establecidos, la autenticación y encriptación de la contraseña \\ \hline
%		
%		\textbf{T12}
%		& Provee acceso directo a terceras partes
%		& 1 & 0 & 0
%		& No provee acceso a terceras partes \\ \hline
%		
%		\textbf{T13}
%		& Se requieren facilidades especiales de entrenamiento
%		& 1 & 3 & 3
%		& El sistema es fácil de usar, pero el usuario requiere entrenamiento en su manejo \\		
%
%		\hline
%	\end{tabular}
%	\caption[Cálculo del Factor de Complejidad Técnica]{Cálculo del Factor de Complejidad Técnica}
%	\label{table:CalcFCT}
%\end{table}
%
%\paragraph{}Después de haber obtenido estos valores se procede a calcular el Factor de Complejidad Técnica ($FCT$).
%\begin{center}
%	{ \fboxsep 12pt
%		\fcolorbox {black}{white}{
%			\begin{minipage}[t]{7cm}
%				$FCT = 0.6 + 0.01 * \sum (Peso (i) * Valor (i))$\\
%				$FCT = 0.6 + 0.01 * (44.5) = 0.6 + 0.445$\\
%				$FCT = 1.045$
%			\end{minipage}
%		}
%	}
%\end{center}
%
%\subsubsection*{FA = Factor de Ambiente }
%\paragraph{}El Factor de Ambiente (FA) está relacionado con las habilidades y el entrenamiento del grupo de desarrollo del proyecto, estos factores tienen impacto en las estimaciones de tiempo. Este cálculo es similar al del Factor de complejidad técnica, se trata de un conjunto de factores que se cuantifican con valores de 0 a 5.
%\paragraph{}En la tabla \ref{table:DetermFA} se muestra el significado y el peso de cada uno de los factores y los valores asignados a cada uno de ellos.
%
%\begin{table}[ht]
%	\begin{tabular}{ | l |p{4.5cm}| l | l | l |p{5.5cm}| }
%		\hline
%		\textbf{Factor}
%		& \textbf{Descripción}
%		& \textbf{Peso}
%		& \textbf{Valor}
%		& \textbf{Peso*Valor}
%		& \textbf{Comentario} \\ \hline
%		
%		\textbf{E1}
%		& Familiaridad con el modelo de proyecto utilizado
%		& 1.5 & 5 & 7.5
%		& Existe familiaridad con el modelo de proyecto utilizado \\ \hline
%		
%		\textbf{E2}
%		& Experiencia en la aplicación
%		& 0.5 & 5 & 2.5
%		& Se tiene experiencia en la aplicación \\ \hline
%		
%		\textbf{E3}
%		& Experiencia en orientación a objetos
%		& 1 & 5 & 5
%		& Se tiene experiencia en la programación orientada a objetos \\ \hline
%		
%		\textbf{E4}
%		& Capacidad del analista líder
%		& 0.5 & 5 & 2.5
%		& El líder tiene la suficiente capacidad para llevar a cabo el desarrollo de la aplicación \\ \hline
%		
%		\textbf{E5}
%		& Motivación
%		& 1 & 5 & 5
%		& Existe alta motivación \\ \hline
%		
%		\textbf{E6}
%		& Estabilidad de los requerimientos
%		& 2 & 4 & 8
%		& Los requerimientos de la aplicación son estables \\ \hline
%		
%		\textbf{E7}
%		& Personal a tiempo parcial
%		& -1 & 2 & -2
%		& Tiempo compartido entre estudio y desarrollo de la aplicación \\ \hline
%		
%		\textbf{E8}
%		& Dificultad del lenguaje de programación
%		& -1 & 2 & -2
%		& Se empleará Java, el cual es un lenguaje de programación común \\
%		
%		\hline
%		\end{tabular}
%		\caption[Determinación del Factor de Ambiente]{Determinación del Factor de Ambiente}
%		\label{table:DetermFA}
%\end{table}
%
%\paragraph{}Con los resultados de la tabla anterior se procede a calcular el $FA$ (Factor de Ambiente) cuyo resultado se expresa en la siguiente ecuación:
%\begin{center}
%	{ \fboxsep 12pt
%		\fcolorbox {black}{white}{
%			\begin{minipage}[t]{9cm}
%				$FA = 1.4 – 0.03 * \sum (Peso (i) * Valor asignado (i)$ \\
%				$FA = 1.4 – 0.03 * (7.5 + 2.5 + 5 + 2.5 + 5 + 8 -2 -2)$ \\
%				$FA = 1.4 – 0.03 * (26.5)$ \\
%				$FA = 1.4 – 0.795$ \\
%				$FA = 0.605$
%				\end{minipage}
%		}
%	}
%\end{center}
%
%\section{Puntos de Casos de Uso Ajustados (PCUA)}
%\paragraph{}Por último y al tomar los valores calculados anteriormente se obtienen los Puntos de Casos de Uso Ajustados (PCUA).
%\begin{center}
%	{ \fboxsep 12pt
%		\fcolorbox {black}{white}{
%			\begin{minipage}[t]{5cm}
%				$PCUA = PCU * FCT * FA$ \\
%				$PCUA = 208 * 1.045 * 0.605$ \\
%				$PCUA = 131.503$
%			\end{minipage}
%		}
%	}
%\end{center}
%
%\section{Estimación del Esfuerzo}
%\paragraph{}Para el cálculo del esfuerzo en hombres/horas se utiliza la siguiente ecuación:
%\begin{center}
%	{ \fboxsep 12pt
%		\fcolorbox {black}{white}{
%			\begin{minipage}[t]{3cm}
%				$E = PCUA * FC$
%			\end{minipage}
%		}
%	}
%\end{center}
%
%Donde:
%\begin{itemize}
%	\item $E$ = Esfuerzo estimado en hombres/horas.
%	\item $PCUA$ = Puntos de Casos de Uso Ajustados.
%	\item $FC$ = Factor de Conversión.
%\end{itemize}
%
%\paragraph{}Para calcular el Factor de Conversión (FC) se analiza el Factor Ambiente y se observa cuáles de los que afectan el factor ambiente (E1…E6) se encuentran por debajo de la media (3) y que el resto (E7, E8) esté por encima de la media (3).
%\paragraph{}Para asignar la cantidad de horas/hombres se analiza que:
%\begin{itemize}
%	\item Si el total es dos o menos se utiliza el factor de conversión de veinte horas/hombre por Punto de Casos de Uso.
%	\item Si el total es tres o cuatro se utiliza como factor de conversión el valor de veinte y ocho horas/hombre por Punto de Casos de Uso.
%	\item Si el total es mayor o igual que cinco se recomienda efectuar cambios en el proyecto ya que existe un alto riesgo de fracaso.
%\end{itemize}
%
%\paragraph{}Al tomar como base lo expresado anteriormente se puede decir que:
%\begin{center}
%	{ \fboxsep 12pt
%		\fcolorbox {black}{white}{
%			\begin{minipage}[t]{8cm}
%				$FC = Horas/hombres * Punto de Casos de Uso$
%			\end{minipage}
%		}
%	}
%\end{center}
%\paragraph{}Al sustituir en la ecuación anterior obtenemos el siguiente resultado:
%\begin{center}
%	{ \fboxsep 12pt
%		\fcolorbox {black}{white}{
%			\begin{minipage}[t]{5cm}
%				$E = PCUA * FC$ \\
%				$E = 131.503 * 20 horas$ \\
%				$E = 2630.06 horas/hombres$
%			\end{minipage}
%		}
%	}
%\end{center}
%
%\paragraph{}La tabla \ref{table:EsfReq} muestra el cálculo del esfuerzo requerido, al tomar como base los porcentajes especificados:
%
%\begin{table}[ht]
%	\centering
%	\begin{tabular}{ | l | l | l | }
%		\hline
%		\textbf{Actividad}
%		& \textbf{Porcentaje}
%		& \textbf{Horas/Hombre} \\ \hline
%		
%		\textbf{Análisis}
%		& 10\% & 657.51 \\ \hline
%		
%		\textbf{Diseño}
%		& 20\% & 1315.03 \\ \hline
%		
%		\textbf{Implementación}
%		& 40\% & 2630.06 \\ \hline
%		
%		\textbf{Pruebas}
%		& 15\% & 986.27 \\ \hline
%		
%		\textbf{Sobrecarga (otras actividades)}
%		& 15\% & 986.27 \\ \hline
%		
%		\textbf{Total}
%		& 100\% & 6575.14 \\
%		
%		\hline
%		\end{tabular}
%		\caption[Esfuerzo requerido para el desarrollo del proyecto]{Esfuerzo requerido para el desarrollo del proyecto}
%		\label{table:EsfReq}
%\end{table}
%
%\paragraph{}Este esfuerzo es el que se requiere para la implementación del sistema. Si se tiene en cuenta que la fase de implementación representa el 40\% del esfuerzo total para desarrollar el software, entonces se puede calcular el esfuerzo total de la siguiente manera:
%\begin{center}
%	{ \fboxsep 12pt
%		\fcolorbox {black}{white}{
%			\begin{minipage}[t]{5cm}
%				$E (Total) = E / 0.4$ \\
%				$E (Total) = 2630.06 / 0.4$ \\
%				$E (Total) = 6575.14$
%			\end{minipage}
%		}
%	}
%\end{center}
%
%\section{Estimación del tiempo de desarrollo del proyecto}
%\begin{center}
%	{ \fboxsep 12pt
%		\fcolorbox {black}{white}{
%			\begin{minipage}[t]{5cm}
%				$TDES (total) =$ $E (total) \over CH (total) $
%			\end{minipage}
%		}
%	}
%\end{center}
%Donde: 
%\begin{itemize}
%	\item $TDES$ = Tiempo de desarrollo
%	\item $CH$ = Cantidad de Hombres = 1
%\end{itemize}
%
%\begin{center}
%	{ \fboxsep 12pt
%		\fcolorbox {black}{white}{
%			\begin{minipage}[t]{6cm}
%				$TDES (total) =$ $E (total) \over CH (total)$ \\
%				$TDES (total) =$ $6575 horas/hombre \over 1 hombre$ \\
%				$TDES (total) = 6575 horas$
%			\end{minipage}
%		}
%	}
%\end{center}
%
%\paragraph{}Si consideramos que el mes tiene 190.6 horas laborales (8 horas de trabajo por 24 días laborables como promedio)
%\begin{center}
%	{ \fboxsep 12pt
%		\fcolorbox {black}{white}{
%			\begin{minipage}[t]{8cm}
%				$$TDES (total) = \frac{6575 horas}{190.6 horas por mes}$$ \\
%				$TDES (total) = 34.5 meses \approx 35 meses$ \\
%				$TDES (total) = \frac{35 meses}{12 meses (1 ano)} = 2.916 anos$
%			\end{minipage}
%		}
%	}
%\end{center}
%\paragraph{}Lo que significa que el tiempo de desarrollo del proyecto es aproximadamente de 2 años y 9 meses.
%
%\section{Estimación del costo de desarrollo del proyecto}
%\begin{table}[ht]
%	\centering
%	\begin{tabular}{ |p{7cm}| l | l | }
%		\hline
%		\textbf{Variable}
%		& \textbf{Fórmula}
%		& \textbf{Valor} \\ \hline
%		
%		\textbf{C (total): costo total}
%		& $C(total)=C(p)+Otros gastos$
%		& 25 871.31 \\ \hline
%		
%		\textbf{C (p): costo del proyecto}
%		& $C (p)= E (total) * CHH$
%		& 25 871.31 \\ \hline
%		
%		\textbf{E (total): esfuerzo total}
%		& 
%		& 6575 \\ \hline
%		
%		\textbf{CHH: costo por hombre hora}
%		& $CHH= K * THP$
%		& 3.9348 \\ \hline
%		
%		\textbf{K: Coeficiente que tiene en cuenta los costos indirectos}
%		& 
%		& 1.50 \\ \hline
%		
%		\textbf{THP: Tarifa Horaria Promedio}
%		& $THP=SP/190.6$
%		& 2.6232 \\ \hline
%		
%		\textbf{SP: salario promedio}
%		& $SP=\sum _{n=i}^{n}\frac{Si}{n}$
%		& 500 \\\hline
%		
%		\textbf{Si: salario de trabajador}
%		& 
%		& 1200 \\
%		
%		\hline
%	\end{tabular}
%	\caption[Estimación total del costo del proyecto]{Estimación total del costo del proyecto}
%	\label{table:EstTotProy}
%\end{table}
%
%\begin{table}[ht]
%	\centering
%	\begin{tabular}{ | l | l | l | }
%		\hline
%		\textbf{Indicadores}
%		& \textbf{Valor}
%		& \textbf{Unidades} \\ \hline
%		
%		\textbf{Esfuerzo}
%		& 6575
%		& horas/hombre \\ \hline
%		
%		\textbf{Tiempo}
%		& 34.5
%		& meses \\ \hline
%		
%		\textbf{Desarrollo}
%		& 25 871.31
%		& MN \\
%		
%		\hline
%	\end{tabular}
%	\caption[Valor de los indicadores del proyecto]{Valor de los indicadores del proyecto}
%	\label{table:ValIProy}
%\end{table}
%
%\paragraph{}El costo estimado de desarrollo del proyecto está dado por la siguiente fórmula:
%\begin{center}
%	{ \fboxsep 12pt
%		\fcolorbox {black}{white}{
%			\begin{minipage}[t]{5cm}
%				$C (Total) = E (Total) * CHH$
%			\end{minipage}
%		}
%	}
%\end{center}
%\paragraph{}Donde:  
%\begin{itemize}
%	\item $C (Total)$: Costo total del proyecto.
%	\item $E (Total)$: Esfuerzo total.
%	\item $CHH$: Costo por hombre-hora.
%	\item $CHH = K * THP$
%\end{itemize}
%\paragraph{}Donde:  
%\begin{itemize}
%	\item $K$: Coeficiente que tiene en cuenta los costos indirectos (1,5 y 2,0).
%	\item $THP$: Tarifa Horaria Promedio. Se calcula al dividir el salario promedio de las personas que trabajan en el proyecto entre 190.6 horas laborables.
%\end{itemize}
%\begin{center}
%	{ \fboxsep 12pt
%		\fcolorbox {black}{white}{
%			\begin{minipage}[t]{8cm}
%				$Salario promedio = \sum \frac{Salario de los desarrolladores}{Cantidad de desarrolladores}$
%			\end{minipage}
%		}
%	}
%\end{center}
%\begin{center}
%	{ \fboxsep 12pt
%		\fcolorbox {black}{white}{
%			\begin{minipage}[t]{5cm}
%				$CHH = 1.5 * 2.623 = 3.9348$
%			\end{minipage}
%		}
%	}
%\end{center}
%\paragraph{}El proyecto cuenta con un desarrollador que tiene como salario 1200 CUP como promedio, por lo que:
%\begin{center}
%	{ \fboxsep 12pt
%		\fcolorbox {black}{white}{
%			\begin{minipage}[t]{4.5cm}
%				$THP = \frac{1200}{190.6} = 6.29590$
%			\end{minipage}
%		}
%	}
%\end{center}
%\begin{center}
%	{ \fboxsep 12pt
%		\fcolorbox {black}{white}{
%			\begin{minipage}[t]{5.5cm}
%				$C (Total) = E (Total) * K * CHH$ \\
%				$C (Total) = 6575 * 6.29590$ \\
%				$C (Total) = 41 395.54 CUP$
%			\end{minipage}
%		}
%	}
%\end{center}
%
%\paragraph{}Si utilizamos el tipo de cambio doméstico 25 CUP x 1.00 CUC y para poder comparar el costo de este proyecto en relación con los existentes en el mercado y que fueron analizados en el capítulo 1 de este documento, tenemos:
%\begin{center}
%	{ \fboxsep 12pt
%		\fcolorbox {black}{white}{
%			\begin{minipage}[t]{5cm}
%				$C (Total) = 1655.82 CUC$
%			\end{minipage}
%		}
%	}
%\end{center}

\section{Beneficios tangibles e intangibles}
\paragraph{}
Los beneficios tangibles son ventajas que se pueden medir en dólares que se acreditan a la organización mediante el uso del sistema de información los ejemplos de beneficios tangibles son un aumento en la velocidad del procesamiento, acceso de otra forma a la información inaccesible, acceso a la información en una forma más oportuna, ventaja por el poder de cálculo de la computadora y las disminuciones en el tiempo de empleado necesario para cumplir las tareas específicas. Aún hay otros, aunque la medición no siempre es fácil, actualmente los beneficios tangibles se pueden medir en términos de ahorros en dólares, recursos o tiempo \cite{clavo_beneficios_2013}.

\subsection{Beneficios tangibles}
\begin{itemize}
	\item El CIM cuenta con una herramienta cuyo costo de desarrollo asciende a $\$$ $411,587.00$ $CUP$ ($\$$ $1655.82$ $CUC$).
	\item Esta herramienta provee a la entidad la capacidad de contar con históricos de movimientos de operaciones de caja.
	\item Se elimina el gasto por concepto de modelaje para las operaciones de caja el cual era pagado en CUC.
	\item Se obtiene un \textit{software Open Source}, integrado a los procesos que se desarrollan en la caja, y se evita a la empresa el invertir en la compra de \textit{software} y la licencia.
\end{itemize}

\subsection{Beneficios intangibles}
\paragraph{}Los beneficios intangibles incluyen mejorar el proceso de toma de decisiones, incrementar la exactitud ser más competitivo con el servicio al cliente, mantener una buena imagen del negocio e incrementar la satisfacción del trabajo para los empleados eliminando las tareas tediosas. Cómo puede juzgar de la lista dada, los beneficios intangibles son sumamente importantes y pueden tener consecuencias transcendentales para el negocio conforme relaciona a las personas fuera y dentro de la organización \cite{clavo_beneficios_2013}.
\begin{itemize}
	\item Se evita la pérdida de información mediante una organización eficiente de estos en un sistema informático.
	\item Se homogeniza la información emitida en la caja con los documentos aprobados por el MFP y la misma puede ser consultada en cualquier momento.
	\item Se pueden realizar arqueos sorpresivos sobre la base de la información con que se cuenta en la aplicación, en ambas monedas.
\end{itemize}

\section{Análisis de costos y beneficios}
\paragraph{}Con la implantación del software propuesto la Dirección de Economía del Centro de Inmunología Molecular contará con una herramienta capaz de humanizar las operaciones de tesorería de caja de la empresa, así como permitirá controlar los datos que se manejan en ella. Si se toma el costo estimado del proyecto $\$$       $11,567.00$ $CUP$ ($\$$ $463.48$ $CUC$) y para su puesta en marcha no se incurrirán en gastos adicionales, esta opción constituye un ahorro de dinero. Además, con el uso de esta herramienta se mantienen actualizados los movimientos de efectivo existente en caja en tiempo real.

\section{Conclusiones parciales}
\begin{itemize}
	\item Los costos que se incurren en el desarrollo del sistema, ascienden a $\$$  $11,567.00$ $CUP$ ($\$$ $463.48$ $CUC$) en un tiempo estimado de 11 meses al trabajar una sola persona en el desarrollo del proyecto.
	\item Se realizaron cuatro casos de prueba y ocho escenarios en dos iteraciones.
\end{itemize}