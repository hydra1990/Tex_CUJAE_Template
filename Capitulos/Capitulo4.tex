% Para todos los capítulos, use la funcionalidad nueva chap{} en vez de chapter{}
% Esto hará que el texto en la izquierda superior de la página sea el mismo que el del capítulo

% Los Nombres de los capítulos se definen en Configuraciones/Administrativo.tex
\chap{\CapCuatro}

\sf
\section{Introducción}
\paragraph{}El diseño del \textit{software} es la última acción de la ingeniería correspondiente dentro de la actividad de modelado, la cual establece una plataforma para la construcción (generación de código y pruebas). Permite al ingeniero de \textit{software} modelar el sistema o producto que se va a construir, al convertir los requisitos del cliente en funcionalidades. Este flujo de trabajo garantiza que no se corran riesgos de construir un sistema inestable, propenso a fallos a la hora de realizar cambios, por muy pequeños que estos sean. Además, es la etapa donde se establece y fomenta la calidad del \textit{software} mediante el modelado de toda la solución \cite{goran_v._rational_2007}.

\section{Diagrama de clases de diseño}
\subsection{Diagrama de clases de diseño paquete configuraci\'{o}n}
Ver Anexos, figura \ref{fig:DCDPktConf}.

\subsection{Diagrama de clases de diseño paquete operaciones}
Ver Anexos, figura \ref{fig:DCDPktOper}.

\subsection{Diagrama de clases de diseño paquete reportes}
Ver Anexos, figura \ref{fig:DCDPktRep}.

\subsection{Diagrama de clases de diseño paquete seguridad}
Ver Anexos, figura \ref{fig:DCDPktSeg}.

\section{Estructuraci\'{o}n en capas}
\paragraph{}La arquitectura estructurada en capas permite que se encapsulen los datos de tal forma que se pueda tratar el código modularmente, sin correr el peligro de que, al modificar un código, deje de ser funcional la aplicación. Esto se logra al abstraer todo el código en diferentes capas, lo que permite además la interoperabilidad de las clases y la reutilización del código.
\paragraph{}El patrón de diseño Modelo Vista Controlador (MVC) es un estilo de arquitectura de \textit{software} que separa los datos de una aplicación, la interfaz de usuario, y la lógica de control en tres componentes distintos.
\paragraph{}Se trata de un modelo muy maduro y que ha demostrado su validez a lo largo de los años en todo tipo de aplicaciones, y sobre multitud de lenguajes y plataformas de desarrollo \cite{noauthor_modelo_2019}.

\begin{figure}[H] %Con el paquete float, se pone la opción [H] para que te salga la imagen donde la quieres
	\centering
	\includegraphics[width=15cm,height=8cm]{Figuras/MVC.png}
	\caption{Representación modelo de diseño arquitectónico Modelo Vista Controlador MCV}
	\label{fig:MVC}
\end{figure}

\paragraph{}A continuación se muestra el uso del patrón de diseño arquitectónico Modelo Vista Controlador en el sistema propuesto.

\begin{figure}[H] %Con el paquete float, se pone la opción [H] para que te salga la imagen donde la quieres
	\centering
	\includegraphics[width=17cm,height=8cm]{Figuras/MVCApp.png}
	\caption{Representación modelo de diseño arquitectónico Modelo Vista Controlador MCV}
	\label{fig:MVCApp}
\end{figure}

\paragraph{}Al combinar la arquitectura en capas con el enfoque MVC, se obtiene una arquitectura dividida en 3 capas, las cuales se tratarán según establece el patrón MVC. A continuación, en la en la Figura Estructuración en Capas se muestra el diagrama de estructuración en capas mediante el enfoque basado en Responsabilidad.
\paragraph{Capa Presentación:}Es la que presenta el sistema al usuario también conocida como interfaz de usuario, comunica la información y captura la información del usuario en un mínimo de proceso. Esta capa se comunica únicamente con la capa de negocio. 
\paragraph{Capa Lógica de Negocio:}En ella residen los programas que se ejecutan, se reciben las peticiones del usuario y se envían las respuestas tras el proceso, es aquí donde se establecen todas las reglas que deben cumplirse. Esta capa se comunica con la capa de presentación, para recibir las solicitudes y presentar los resultados, y con la capa de acceso a datos, para solicitar al gestor de base de datos almacenar o recuperar datos de él. 
\paragraph{Capa Acceso a Datos:}En ella residen los datos. Está formada por uno o más gestor de bases de datos que realiza todo el almacenamiento de datos, reciben solicitudes de almacenamiento o recuperación de información desde la capa de negocio.

\begin{figure}[H] %Con el paquete float, se pone la opción [H] para que te salga la imagen donde la quieres
	\centering
	\includegraphics[width=16cm,height=10cm]{Figuras/Estructura_capas.png}
	\caption{Estructuración en capas}
	\label{fig:Capas}
\end{figure}

\section{Patrones de diseño}
\paragraph{}Un patrón de diseño es una solución repetible a problemas típicos y recurrentes en el diseño del \textit{software}. Son soluciones basadas en la experiencia y que se ha demostrado que funcionan. No son un diseño terminado que puede traducirse directamente a código, sino más bien una descripción sobre cómo resolver el problema, la cual puede ser utilizada en diversas situaciones. Eric Gamma define los patrones de diseño como:
\paragraph{}“Un patrón de diseño es una descripción de clases y objetos comunicándose entre sí adaptada para resolver un problema de diseño general en un contexto particular” \cite{gamma_reutilizacion_2015}
\paragraph{}Por otra parte, Christopher Alexander describe a los patrones de conjunto con la Integridad Conceptual.
\paragraph{}"…  Cada patrón describe un problema que ocurre una y otra vez en nuestro entorno, para describir después el núcleo de la solución a ese problema, de tal manera que esa solución pueda ser usada más de un millón de veces sin hacerlo ni siquiera dos veces de la misma forma …" \cite{alexander_pattern_1997}.

\subsection{Fachada}
\paragraph{}El patrón FACADE simplifica el acceso a un conjunto de clases al proporcionar una única clase que todos utilizan para comunicarse con dicho conjunto de clases \cite{gamma_reutilizacion_2015}.
\subsubsection*{Ventajas}
\begin{itemize}
	\item Los clientes no necesitan conocer las clases que hay tras la clase FACADE.
	\item Se pueden cambiar las clases “ocultadas” sin necesidad de cambiar los clientes. Sólo hay que realizar los cambios necesarios en FACADE.
\end{itemize}

\paragraph{}En la aplicación se utiliza para facilitar el acceso a los datos, y la interacción con los mismos, como se muestra en la figura \ref{fig:Fachada}.

\begin{figure}%[H] %Con el paquete float, se pone la opción [H] para que te salga la imagen donde la quieres
	\centering
	\includegraphics[width=18cm,height=10cm]{Figuras/Fachada.png}
	\caption{Uso del patrón fachada}
	\label{fig:Fachada}
\end{figure}

\section{Principios de diseño}
\paragraph{}Para ayudar a desarrollar programas robustos, mantenibles y que se puedan modificar, existen varios principios de diseño que ayudan a desarrollar software de este tipo.
\paragraph{}Aunque la mayoría son aplicables a la programación orientada a objetos - el paradigma más extendido - algunos de ellos se pueden aplicar a otros tipos de programación \cite{rubenfa_doce_2014}.

\subsection{Interfaz de usuario}
\paragraph{}En todo desarrollo de \textit{software} la interfaz con la que se relaciona el usuario es de vital importancia para la aplicación, en este caso ha sido creada para aquel grupo de usuarios que tiene poca experiencia en el campo de la informática. El diseño desarrollado es en ventana, para seguir la misma estructura que la mayoría de las aplicaciones, es estable y uniforme. El diseño se rige fundamentalmente en color hueso, sin grandes pretensiones, con una imagen de fondo del Centro de Inmunología Molecular, en la misma aparece el nombre de la aplicación y su logo, en la parte inferior derecha se muestra el nombre del usuario que se encuentra al trabajar en el sistema, así como la fecha del día corriente. Se utilizan marcos rectangulares para los menús y submenús bien definidos, así como botones pequeños con una ayuda \textit{in situ} que permite al usuario ubicarse en la acción que tiene implícita cada uno. Para acceder al sistema el usuario debe ingresar su nombre de usuario y contraseña. Se utiliza una misma tipografía, forma y estilo en todas las ventanas, así como la simplicidad y consistencia de las mismas. 
\paragraph{}A continuación, se muestra la figura \ref{fig:AppMain}.

\begin{figure}[H] %Con el paquete float, se pone la opción [H] para que te salga la imagen donde la quieres
	\centering
	\includegraphics[width=17cm,height=10cm]{Figuras/APPMain.png}
	\caption{Interfaz gráfica de la aplicación}
	\label{fig:AppMain}
\end{figure}

\subsection{Formato de salida de los reportes}
\paragraph{}Todo sistema informático requiere salidas de la información que procesa para mostrar los resultados. En nuestro caso están encaminados a obtener información sobre las operaciones que procesa la caja y están basados en las normas emitidas por el Ministerio de Finanzas y Precios. Para la gestión de los reportes de la aplicación se usa \textit{IReport} debido a los beneficios que ofrece en cuanto a la generación de reportes y la integración con el lenguaje de programación que se eligió. El formato de salida de los reportes que se generan está en correspondencia con lo establecido en una norma contable\cite{noauthor_resolucion_2007}.
\paragraph{}A continuación, se muestra la estructura estándar que siguen los reportes de la aplicación:
\begin{itemize}
	\item Encabezado con nombre de la empresa.
	\item Encabezado con nombre del reporte.
	\item Fecha de emisión.
	\item Datos específicos del reporte.
\end{itemize}

\paragraph{}Para ilustrar lo anterior, se muestran ejemplos del formato de salida de los reportes en la figura \ref{fig:ReporteAnticipo} y la figura \ref{fig:ReporteControlAnticipo}.

\begin{figure}%[H] %Con el paquete float, se pone la opción [H] para que te salga la imagen donde la quieres
	\centering
	\includegraphics[width=17cm,height=10cm]{Figuras/ReporteAnticipo.png}
	\caption{Formato de salida del reporte Anticipo y Liquidación de Gastos de Viaje}
	\label{fig:ReporteAnticipo}
\end{figure}

\begin{figure}%[H] %Con el paquete float, se pone la opción [H] para que te salga la imagen donde la quieres
	\centering
	\includegraphics[width=17cm,height=4cm]{Figuras/ReporteControlAnticipos.png}
	\caption{Formato de salida del reporte Control de Anticipos a Justificar}
	\label{fig:ReporteControlAnticipo}
\end{figure}

\subsection{Principios de codificación}
\paragraph{}Camel Case es un principio de diseño perteneciente a la administración de configuración de cambios que establece un estándar para los nombres de las clases y los métodos y atributos de un sistema.
\paragraph{}Para la implementación de la aplicación se utilizó el principio \textit{Lower Camel Case} (figura \ref{fig:LowerCamelCase}) para los nombres de los atributos y operaciones, mientras que \textit{Upper Camel Case} (figura \ref{fig:UpperCamelCase}) fue utilizado al nombrar las clases de la aplicación.

\begin{figure}[H] %Con el paquete float, se pone la opción [H] para que te salga la imagen donde la quieres
	\centering
	\includegraphics[width=14cm,height=4cm]{Figuras/LowerCamelCase.png}
	\caption{Principio de diseño \textit{Lower Camel Case}}
	\label{fig:LowerCamelCase}
\end{figure}

\begin{figure}[H] %Con el paquete float, se pone la opción [H] para que te salga la imagen donde la quieres
	\centering
	\includegraphics[width=8cm,height=5cm]{Figuras/UpperCamelCase.png}
	\caption{Principio de diseño \textit{Upper Camel Case}}
	\label{fig:UpperCamelCase}
\end{figure}

%\paragraph{}A continuación, en la figura se muestra el uso del patrón de diseño arquitectónico Modelo Vista Controlador (MCV).
%
%\begin{figure}[H] %Con el paquete float, se pone la opción [H] para que te salga la imagen donde la quieres
%	\centering
%	\includegraphics[width=18cm,height=10cm]{Figuras/MVCApp.png}
%	\caption{Aplicación de MVC en el paquete reportes}
%	\label{fig:MVCApp}
%\end{figure}

\section{Ayuda}
\paragraph{}El sistema cuenta con una ayuda simple que permite al usuario navegar por sus opciones, al facilitar su uso.
\paragraph{}A continuación, se muestra una pantalla donde hace referencia a la ayuda del sistema.

\begin{figure}[H] %Con el paquete float, se pone la opción [H] para que te salga la imagen donde la quieres
	\centering
	\includegraphics[width=18cm,height=10cm]{Figuras/Ayuda.png}
	\caption{Pantalla principal ayuda}
	\label{fig:Ayuda}
\end{figure}

\section{Tratamiento de errores}
\begin{figure}[H] %Con el paquete float, se pone la opción [H] para que te salga la imagen donde la quieres
	\centering
	\includegraphics[width=7cm,height=3cm]{Figuras/ErrorAuth.png}
	\caption{Fallo en autenticación}
	\label{fig:ErrorAuth}
\end{figure}

\paragraph{}La información de los errores que se pueden cometer tanto en la introducción de datos como en la validación de los mismos garantiza que la información introducida en el sistema sea lo más fiable posible. A continuación, se muestran algunas pantallas donde se expone el tratamiento de algunas validaciones implementadas en la aplicación.
\paragraph{}La figura \ref{fig:ErrorBill} muestra cómo se implementa la validación para que el usuario no pueda utilizar una cantidad de billetes mayor a la existente en la base de datos.

\begin{figure}[H] %Con el paquete float, se pone la opción [H] para que te salga la imagen donde la quieres
	\centering
	\includegraphics[width=14cm,height=11cm]{Figuras/ErrorBill.png}
	\caption{Error en validación de cantidad de billetes}
	\label{fig:ErrorBill}
\end{figure}

\paragraph{}La figura \ref{fig:CancelDoc} muestra el mensaje del sistema cuando se pretende cancelar un documento, ya que esta acción es irreversible. Nótese el cambio de color de los documentos cancelados con respecto al resto y que no desaparece de la pantalla, para que sea posible llevar el récord de la numeración y el control documental.

\begin{figure} %Con el paquete float, se pone la opción [H] para que te salga la imagen donde la quieres
	\centering
	\includegraphics[width=14cm,height=11cm]{Figuras/CancelDoc.png}
	\caption{Confirmar cancelación de un documento}
	\label{fig:CancelDoc}
\end{figure}

\section{Diseño de la base de datos}
\subsection{Modelo lógico de datos}
\paragraph{}A continuación, se muestran las clases persistentes necesarias para implementar las funcionalidades de la aplicación propuesta. El diagrama de clases persistentes o modelo lógico de datos, sirve de punto de partida para el diseño de la base de datos del sistema.
\paragraph{}Se explica de una forma más detallada como, con la ayuda de Hibernate se realizan estas acciones:
\begin{itemize}
	\item Por cada entidad, que son las clases estereotipadas como “Entity”, se crea una tabla en la base de datos con el mismo nombre. La herramienta genera de forma automática la llave para cada tabla, identificada como id (tipo de dato auto numérico).
	\item Por cada relación de asociación de uno a uno (1:1) se generará en el extremo de la	relación donde se haya especificado la agregación una llave foránea que se	corresponderá con la llave del otro extremo.
	\item Por cada relación de asociación de uno a muchos (1:M) se genera en la que	corresponde al extremo mucho (M) un atributo que funciona como llave foránea que	coincide con la llave (campo id) del extremo uno (1).
	\item Por cada relación de asociación de muchos a muchos (M:M), se crea una tabla	intermedia que tendrá como campos los identificadores de las entidades	relacionadas, y se comportarán como llaves foráneas, mientras que su campo llave	lo ocupará un atributo id con el objetivo de evitar la complejidad de las llaves	compuestas.
	\item Por cada asociación de herencia, se genera por cada clase involucrada una tabla. En las tablas correspondientes a las entidades hijas se crea un atributo que se corresponde con la llave en la tabla generada a partir de la entidad padre
\end{itemize}

\begin{landscape}
\begin{figure}[H] %Con el paquete float, se pone la opción [H] para que te salga la imagen donde la quieres
	\centering
	\includegraphics[width=25cm,height=17cm]{Figuras/ModLogBD.png}
	\caption{Modelo lógico de datos}
	\label{fig:ModLogBD}
\end{figure}

\subsection{Modelo físico de datos}
\paragraph{}El modelo físico de los datos constituye la representación física del modelo de clases persistentes visto anteriormente. Contiene el conjunto de tablas que conforman la base de datos (ver Anexos, figura \ref{fig:ModFisBD}.

\begin{figure}[H] %Con el paquete float, se pone la opción [H] para que te salga la imagen donde la quieres
	\centering
	\includegraphics[width=25cm,height=14cm]{Figuras/ModFisBD.png}
	\caption{Modelo físico de datos}
	\label{fig:ModFisBD}
\end{figure}

\end{landscape}

\subsection{Diagrama de despliegue}
\paragraph{}El diagrama de despliegue es un modelo de objetos que describe la distribución física del sistema mediante nodos. Cada uno de ellos representa un recurso de cómputo dedicado a una función determinada. Este modelo da la correspondencia entre la arquitectura de software y la del sistema hardware \cite{sparxsystems_diagrama_nodate}.
\paragraph{}A continuación se presenta el diagrama de despliegue del sistema propuesto con los requerimientos mínimos de \textit{software} y \textit{hardware} para cada nodo.

\begin{figure}[H] %Con el paquete float, se pone la opción [H] para que te salga la imagen donde la quieres
	\centering
	\includegraphics[width=16cm,height=12cm]{Figuras/DDespliegue.png}
	\caption{Diagrama de despliegue}
	\label{fig:DDespliegue}
\end{figure}

\subsubsection*{Descripción de cada nodo}
\paragraph{Nodo PC Cliente:}Tiene como propósito realizar las peticiones al nodo Servidor de Base de Datos. Constituye una generalización de las diferentes estaciones de trabajo donde se va a ejecutar la aplicación. Es donde el usuario del sistema hará uso de los servicios ofrecidos por el mismo. 
\paragraph{Nodo Servidor de Base de datos:}Su propósito es almacenar la información procesada por la aplicación. 
\paragraph{Nodo Impresora:}Su propósito es imprimir las informaciones que brinda el sistema a través de los reportes.

\section{Conclusiones parciales}
\begin{itemize}
	\item Se aplicó el patrón de diseño “Fachada” al diseño de las clases
	\item La ayuda del sistema guía al usuario a través de todas las funcionalidades implementadas en la aplicación
\end{itemize}