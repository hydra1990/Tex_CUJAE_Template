% Opinión del tutor de la investigación
\clearpage
\setstretch{1.3} % Resetear el espaciado a 1.3 para el cuerpo del texto (si cambia)
\addtotoc{Opinión del tutor del trabajo de diploma}  % Adiciona la entrada "Síntesis" a los Contenidos

\thispagestyle{empty}
\section*{Opinión del tutor del trabajo de diploma}
\textbf{Título: \myTitle}

Autor: \myAuthorName
	
\paragraph{}El tutor del presente Trabajo de Diploma considera que durante su ejecución el estudiante mostró las cualidades que a continuación se detallan.
\paragraph{}<Aquí el tutor debe expresar cualitativamente su opinión y medir (usando la escala: muy alta, alta, adecuada) entre otras las cualidades siguientes: 
\begin{itemize}
	\item Independencia 
	\item Originalidad 
	\item Creatividad             
	\item Laboriosidad 
	\item Responsabilidad
\end{itemize}
	
\paragraph{}Además, debe evaluar la calidad científico-técnica del trabajo realizado (resultados y documento) y expresar su opinión sobre el valor de los resultados obtenidos (aplicación y beneficios)
\paragraph{}Por todo lo anteriormente expresado considero que el estudiante está apto para ejercer como Ingeniero Informático; y propongo que se le otorgue al Trabajo de Diploma la calificación de 5 Excelente. Además, consideramos que los resultados del presente trabajo poseen valor científico para ser publicados.  

\begin{tabular}{ l l l }
	\myTutorOne		& \hspace{20 mm} & \myTutorTwo \\
		& \vspace{10 mm} &  \\
	\rule{13em}{0.6pt} 						& \hspace{20 mm} & \rule{11em}{0.5pt} \\
	Instructor		& \hspace{20 mm} & Instructor \\
\end{tabular}

\paragraph{}Fecha:  \rule{2em}{0.5pt} de \rule{10em}{0.5pt} del \rule{3em}{0.5pt}.